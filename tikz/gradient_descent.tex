% !TEX TS-program = xelatex
% !TEX encoding = UTF-8 Unicode

\documentclass[11pt,tikz,border=1]{standalone}
\usepackage[default,mdseries=Light,bfseries=Medium,path=fonts]{cjkfonts}
\usetikzlibrary{calc,positioning,arrows.meta,shapes.geometric,shapes.misc}
\usepackage{pgfplots}

\begin{document}
\begin{tikzpicture}
  
  \begin{axis}[
    name=before,
    width=12cm,
    height=9cm,
    title={梯度下降},
    xlabel={$x$},
    ylabel=\empty,
    xtick distance=0.5,
    ytick distance=0.5,
    xticklabel style={font=\footnotesize,
      /pgf/number format/precision=1,
      /pgf/number format/fixed,
      /pgf/number format/fixed zerofill},
    yticklabel style={font=\footnotesize,
      /pgf/number format/precision=1,
      /pgf/number format/fixed,
      /pgf/number format/fixed zerofill},
    xmin=-2.0,
    xmax=2.0,
    ymin=-2.0,
    ymax=2.0,
    legend pos=south east
    ]

    \addplot[
      blue,
      dashed,
      thick,
      samples=201,
      domain=-2:2
    ]{
      (1/2) * x^2
    };
    \addlegendentry{$f(x) = \frac{1}{2}x^2$}
        
    \addplot[
      green!60!gray,
      thick,
      samples=201,
      domain=-2:2
    ]{
      x
    };
    \addlegendentry{$f'(x) = x$}
    
    \coordinate (o) at (axis cs:0,0);
    \coordinate (o1) at (axis cs:0,1);
    \coordinate (a) at (axis cs:-1,0.5);
    \coordinate (a1) at (axis cs:-1,0);
    \coordinate (b) at (axis cs:1,0.5);
    \coordinate (b1) at (axis cs:1,0);
    
    \draw (o) to (o1);
    \draw (a) to (a1);
    \draw (b) to (b1);
    
    \node[above] at (o1) {
      \footnotesize
      \begin{tabular}{l}
        全局最小值在 $x = 0$ 处\\
        因为 $f'(x) = 0$,梯度\\
        下降在这里停止
      \end{tabular}
    };
    \node[below] at (a1) {
      \footnotesize
      \begin{tabular}{l}
        对于 $x < 0$,我们有 $f'(x) < 0$\\
        所以我们能通过往右移动\\
        来减小 $f$
      \end{tabular}
    };
    \node[below] at (b1) {
      \footnotesize
      \begin{tabular}{l}
        对于 $x > 0$,我们有 $f'(x) > 0$\\
        所以我们能通过往左移动\\
        来减小 $f$
      \end{tabular}
    };
    
  \end{axis}
    
\end{tikzpicture}
\end{document}
